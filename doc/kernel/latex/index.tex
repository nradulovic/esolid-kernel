\hypertarget{index_kernel_specs}{}\subsection{e\-Solid R\-T Kernel Features}\label{index_kernel_specs}
\hypertarget{index_spec_source}{}\subsubsection{Source code}\label{index_spec_source}
The source code of the kernel and all of it's ports are published under {\bfseries free software license}, which guarantees end users (individuals, organizations, companies) the freedoms to use, study, share (copy), and modify the software.

The G\-P\-L grants the recipients of a computer software the rights of the Free Software Definition (written by Richard Stallman) and uses copyleft to ensure the freedoms are preserved whenever the work is distributed, even when the work is changed or added to. The G\-P\-L is a copyleft license, which means that derived works can only be distributed under the same license terms.

For more details visit\-: \href{https://gnu.org/licenses/gpl.html}{\tt https\-://gnu.\-org/licenses/gpl.\-html}\hypertarget{index_spec_api}{}\subsubsection{Consistent Application Programming Interface}\label{index_spec_api}
All objects declared in Application Programming Interface are following these naming rules\-:
\begin{DoxyItemize}
\item All objects except macros are using {\ttfamily Camel\-Case} style names
\item All functions, structures and unions are prefixed with\-: {\ttfamily es}
\item All typedef-\/ed types are prefixed with\-: {\ttfamily es} and postfixed with\-: {\ttfamily \-\_\-\-T}
\item All macro names are in {\ttfamily U\-P\-P\-E\-R\-C\-A\-S\-E} style, words are delimited by underscore sign
\item All macro names are prefixed with\-: {\ttfamily E\-S\-\_\-}
\item All Global variables are prefixed with\-: {\ttfamily g}
\end{DoxyItemize}

All A\-P\-I objects are named following this convention\-: {\ttfamily es$<$group$>$$<$action$>$$<$suffix$>$()}


\begin{DoxyItemize}
\item Group\-:
\begin{DoxyItemize}
\item {\ttfamily Kern} -\/ General Kernel services
\item {\ttfamily Thd} -\/ Thread management
\item {\ttfamily Thd\-Q} -\/ Thread Queue management
\item {\ttfamily Sched} -\/ Scheduler invocation
\item {\ttfamily Sched\-Rdy} -\/ Scheduler Ready Thread Queue management
\end{DoxyItemize}
\item Suffix\-:
\begin{DoxyItemize}
\item {\ttfamily none} -\/ normal A\-P\-I object
\item {\ttfamily I} -\/ I class -\/ Regular Interrupts are locked
\end{DoxyItemize}
\end{DoxyItemize}

All Port Interface objects are named using the rules stated above with certain differences\-:
\begin{DoxyItemize}
\item All functions, structures and unions are prefixed with\-: {\ttfamily port}
\item All macro names are prefixed with\-: {\ttfamily P\-O\-R\-T\-\_\-}
\end{DoxyItemize}\hypertarget{index_spec_preempt}{}\subsubsection{Preemptive multi-\/threading}\label{index_spec_preempt}
e\-Solid R\-T Kernel uses a {\bfseries preemptive scheduler}, which has the power to preempt, or interrupt, and later resume, other threads in the system. The scheduler always runs ready thread with the highest priority.\hypertarget{index_spec_rr}{}\subsubsection{Round-\/\-Robin scheduling}\label{index_spec_rr}
Round-\/\-Robin scheduling is very {\bfseries simple algorithm} to implement and it is starvation free$\ast$$\ast$. It employs time-\/sharing, giving to each thread a time slice or {\ttfamily quantum}. Processor's time is shared between a number of threads, giving the illusion that it is dealing with these threads {\bfseries concurrently}. This scheduling is only used when there are two or more threads of the same priority ready for execution.\hypertarget{index_spec_deterministic}{}\subsubsection{Deterministic}\label{index_spec_deterministic}
Majority of algorithms used in e\-Solid R\-T Kernel implementation are belonging to Constant Time Complexity$\ast$$\ast$ category. Constant Time {\ttfamily O(1)} functions needs fixed amount of time to execute an algorithm. In other words the execution time of Constant Time Complexity functions does not depend on number of inputs. For more information see \hyperlink{time_complexity}{Time complexity}.\hypertarget{index_spec_cfg}{}\subsubsection{Configurable}\label{index_spec_cfg}
The kernel provides two configuration files kernel\-\_\-cfg.\-h and \hyperlink{cpu__cfg_8h}{cpu\-\_\-cfg.\-h} which can be used to tailor the kernel to application needs.

In addition, the kernel implements a number of hooks which can alter or augment the behavior of the kernel or applications, by intercepting function calls between software components.\hypertarget{index_spec_portable}{}\subsubsection{Portable}\label{index_spec_portable}
During the design stage of the kernel a special attention was given to achieve high portability of the kernel. Some data structures and algorithms are tailored to exploit new hardware features.\hypertarget{index_spec_static}{}\subsubsection{Static object allocation}\label{index_spec_static}
All objects used in e\-Solid R\-T Kernel can be statically allocated. There is no need to use any memory management functionality which makes it very easy to verify the application.\hypertarget{index_spec_unlimited}{}\subsubsection{Unlimited number of threads}\label{index_spec_unlimited}
e\-Solid R\-T Kernel allows applications to have any number of threads. The only limiting factors for the maximum number of threads are the amount of R\-A\-M and R\-O\-M memory capacity and required processing time.\hypertarget{index_spec_prio}{}\subsubsection{Up to 256 thread priority levels}\label{index_spec_prio}
Each thread has a defined priority. Lowest priority level is 0, while the highest available level is configurable. If Round-\/\-Robin scheduling is used then multiple threads can be in the same priority level. If Round-\/\-Robin scheduling is disabled then each thread must have unique priority level. The priority sorting algorithm has constant time complexity which means it always executes in the same time period regardles of the levels of priority used.\hypertarget{index_spec_tickless}{}\subsubsection{Tickless idle}\label{index_spec_tickless}
Classic kernel architectures periodically interrupted C\-P\-U at a predetermined frequency — 100 Hz, 250 Hz, or 1000 Hz, depending on the application needs. Known as the system timer tick, the kernel performed this interrupt regardless of the power state of the C\-P\-U. Therefore, even an idle C\-P\-U was responding to up to 1000 of these requests every second. On systems that implemented power saving measures for idle C\-P\-Us, the timer tick prevented the C\-P\-U from remaining idle long enough for the system to benefit from these power savings.

The e\-Solid R\-T kernel runs in tickless idle\-: that is, it replaces the old periodic timer interrupts with on-\/demand interrupts. Therefore, idle C\-P\-Us are allowed to remain idle until a new task is queued for processing, and C\-P\-Us that have entered lower power states can remain in these states longer.\hypertarget{index_spec_errchk}{}\subsubsection{Error checking}\label{index_spec_errchk}
All e\-Solid software is using design methods very similar to approaches of contract programming$\ast$$\ast$ paradigm for software design. The contract programming prescribes that Application Programming Interface should have formal, precise and verifiable specifications, which extend the ordinary definition of abstract data types with preconditions and postconditions. These specifications are referred to as \char`\"{}contracts\char`\"{}. The contract for each method will normally contain the following pieces of information\-:


\begin{DoxyItemize}
\item Acceptable and unacceptable input values
\item Return values and their meanings
\item Error and exception condition values that can occur during the execution
\item Side effects
\item Preconditions
\item Postconditions
\item Invariants
\end{DoxyItemize}

The contract validations are done by {\bfseries assert} macros. They have the responsibility of informing the programmer when a contract can not be validated.\hypertarget{index_spec_prof}{}\subsubsection{Profiling}\label{index_spec_prof}
\begin{DoxyNote}{Note}
This feature is not implemented 
\end{DoxyNote}
